\documentclass{report}
\usepackage[a4paper]{geometry}


\usepackage{graphicx}
\usepackage{booktabs}
\usepackage{amsthm, amsmath, amssymb, amsfonts}
\usepackage{float, subcaption}

\usepackage{fontspec}
\usepackage{polyglossia}
\setmainlanguage[Script=Cyrillic]{serbian}
\setotherlanguage{english}
\setmainfont{Times New Roman}

\theoremstyle{plain}
\newtheorem{thm}{Теорема}
\newtheorem{cor}{Последица}
\theoremstyle{definition}
\newtheorem{defn}{Дефиниција}

\begin{document}

\title{Примене Теорије Група}

\author{Даниел Силађи}

\maketitle

\tableofcontents

\chapter{Мотивација, групе и симетрије}
У свакодневном животу се често дешава да се сусрећемо са предметима за које
кажемо да су ''симетрични''. Шта је заправо симетрија? На пример, можемо рећи да је
неки објекат симетричан ако ''изгледа исто'' кад га гледамо са различитих тачака гледишта.\\
Формалније, симетрија неког објекта је нека трансформација у простору која пресликава
тај објекат на њега самог. Временом, људи су приметили да све овакве трансформације
имају још нека својства, карактеристична за све трансформације:
\begin{enumerate}
  \item Трансформације симетрије су бијективне, тј. за сваку трансформацију постоји инверзна трансформација која ''поништава'' њен ефекат, и доводи објекат у почетно стање. На пример, у случају ротације за угао $\varphi$ око неке осе, инверзна трансформација је ротација за угао $-\varphi$, односно $2\pi-\varphi$ око исте те осе. Неке трансформације, попут осне симетрије могу бити и саме себи инверзне.
  \item Комбиновањем (композицијом) две трансформације такође добијамо трансформацију. На пример, композиција две ротације (са заједночком осом ротације) је опет ротација, композиција две симетрије може бити транслација или ротација, ...
  \item Сваки објекат има једну (тривијалну) симетрију, идентичко пресликавање, које слика сваку тачку у њу саму.
\end{enumerate}
Касније ћемо видети да скуп трансформација (и уопште било каквих апстрактних математичких објеката) са оваквим својствима чини \emph{групу} (групу трансформација у овом случају).
\section{Симетрије раванских фигура}
За почетак, кренимо од неких једноставнијих фигура, правилних многоуглова. Опет, од њих је најједноставнији (једнакостранични) троугао.\\
\begin{figure}[h]
\centering
\includegraphics[width=0.5\textwidth]{placeholder}
\caption{Симетрије једнакостраничног троугла}
\end{figure}\\
Као што видимо, постоји 6 трансформација симетрије: \begin{itemize}
                                                         \item Ротације око центра троугла: $r$ (за $\pi/3$), $r\circ r = r^2$ (за $2\pi/3$) и $r\circ r \circ r = r^3 = e$ (за $2\pi$, односно $0$ радијана - идентичко пресликавање)
                                                         \item Осне симетрије $s_1$, $s_2$ i $s_3$ у односу на праве $\ell_1$, $\ell_2$, $\ell_3$. Приметимо да опет важи $s_i \circ s_i = s_i^2 = e$, за $i\in \lbrace 1, 2, 3 \rbrace$, али и $s_1 s_2 = r^2$, ...
                                                       \end{itemize}
Слично се дешава и код квадрата и правилног шестоугла, само са више оса симетрије и већим степеном ротационе симетрије:\\
\begin{figure}[h]
\centering
\begin{subfigure}{0.4\textwidth}
\includegraphics[width=\textwidth]{placeholder}
\end{subfigure}
~
\begin{subfigure}{0.4\textwidth}
\includegraphics[width=\textwidth]{placeholder}
\end{subfigure}
\caption{Simetrije kvadrata i pravouglog \v sestougla}
\end{figure}\\
Са друге стране, постоје и примери код којих се јавља и транслациона симетрија. Јасно је да то морају бити бесконачни цртежи код којих се један или више основних елементата понављају по једној (\emph{frieze}) или две димензије (\emph{wallpaper}). Интересантно је да се обе врсте оваквих цртежа могу потпуно класификовати у зависности од симетрија које поседују. Тако имамо $7$ frieze-група и $17$ wallpaper-група трансформација.
\begin{figure}[h]
\centering
\begin{subfigure}{0.4\textwidth}
\includegraphics[width=\textwidth]{placeholder}
\end{subfigure}
~
\begin{subfigure}{0.4\textwidth}
\includegraphics[width=\textwidth]{placeholder}
\end{subfigure}
\caption{Примери frieze цртежа}
\end{figure}\\

\begin{figure}[h]
\centering
\begin{subfigure}{0.4\textwidth}
\includegraphics[width=\textwidth]{placeholder}
\end{subfigure}
~
\begin{subfigure}{0.4\textwidth}
\includegraphics[width=\textwidth]{placeholder}
\end{subfigure}
\caption{Примери wallpaper-а}
\end{figure}

\section{Симетрије тродимензионалних тела}
У три димензије, добро су познате симетрије правилних полиедара (тетраедра, коцке, октаедра и икосаедра), али су значајне и групе симетрија молекула и кристала (кристалографске групе), које се такође могу комплетно класификовати, као у дводимензионалном случају.

\begin{figure}[h]
\centering
\begin{subfigure}{0.3\textwidth}
\includegraphics[width=\textwidth]{placeholder}
\caption{Молекул}
\end{subfigure}
~
\begin{subfigure}{0.3\textwidth}
\includegraphics[width=\textwidth]{placeholder}
\caption{Молекул}
\end{subfigure}
~
\begin{subfigure}{0.3\textwidth}
\includegraphics[width=\textwidth]{placeholder}
\caption{Кристална решетка}
\end{subfigure}
\end{figure}

\section{Симетрије у физици}
%Simetrije u klasicnoj i kvantnoj mehanici, npr lorencova grupa i resenja sredingerove jednacine za vodonikov atom. Lijeve grupe.
Коначно, симетрије нашег простора (Poincaré-ова и Lorentz-ова група у специјалној теорији релативности) произилазе из свих квантних теорија поља, док је сам Стандардни Модел (најприхваћенија таква теорија) базиран на унутрашњим симетријама групе $SU(3)\times SU(2)\times U(1)$ (касније ће бити бити објашњено шта то значи).

\chapter{Увод у теорију група}
\section{Основне дефиниције}
%Definicija grupe, red grupe, red elementa, podgrupe.
Као што је већ било речено, елементи неке групе не морају нужно да буду трансформације, већ елементи произвољног скупа $G$, за које смо дефинисали операцију ''множења'' (формално: операцију групе) која задовољава следеће особине:
\begin{enumerate}
  \item Ако $a$ и $b$ припадају $G$, онда и њихов производ, $ab$ припада $G$
  \item Операција множења је асоцијативна, односно важи $a(bc) = (ab)c$
  \item $G$ садржи \emph{јединични елемент} $e$, за који важи $ae = ea = a$, за свако $a\in G$
  \item За свако $a\in G$ постоји $b\in G$, за које важи $ab = ba = e$. Такав $b$ се зове \emph{инверзни елемент} за $a$, и обележава се са $b^{-1}$
\end{enumerate}
Иако операцију групе често називамо ''множењем'', она заправо и може а и не мора то да буде:
\begin{itemize}
  \item Скуп рационалних (или реалних) бројева без $0$ чини групу у односу на множење (у уобичајеном смислу)
  \item Скуп целих (али не и природних!) бројева чини групу у односу на сабирање
  \item Скуп свих инвертибилних (њихова детерминанта је различита од $0$) квадратних матрица димензија $n\times n$ чини групу, а операција групе је множење матрица.
\end{itemize}
Али, чак и у овим примерима нисмо причали о апстрактим групама, него о њиховим конкретним примерима, реализацијама. Структура неке апстрактне групе је задата искључиво дефинисањем операције множења сваког уређеног пара елемената, било набрајањем или на неки други начин, али без позивања на ''природу'' тих елемената.\\
Приметимо да операција множења не мора бити комутативна (нпр код множења матрица), али ако јесте, односно ако за свако $a, b\in G$ важи $ab = ba$, онда је група $G$ комутативна или \emph{Абелова}.\\
Број елемената у групи назива се \emph{ред групе}.\\
Ако одаберено неки елемент $a$ групе $G$, можемо га помножити са самим собом и добити производ $aa$, који ћемо обележавати са $a^2$. У општем случају, производ $$\underbrace{aa...a}_{n\text{ пута}}$$ обележавамо са $a^n$. Слично, можемо дефинисати и негативне степене $a$:
$$a^{-n} = (a^{-1})^n = (a^n)^{-1}$$
Ако су сви степени $a$ различити, кажемо да је $a$ \emph{бесконачног реда}. Иначе, исписивањем степена $a$, наићи ћемо на два природна броја $r$ и $s$, $r>s$, за које важи
$$a^r = a^s.$$
Множењем обе стране једнакости са $a^{-s}$, добијамо
$$a^{r-s} = a^0 = e, r-s>0.$$
Нека је $n$ најмањи број за који је $a^n = e$, $n>0$. Тада је $n$ \emph{ред елемента $a$}.

Primeri: brojevi (sta jeste i sta nije), transformacije, simetrije
Klasifikacija grupa sa 2, 3, 4 elementa, primer da 1 grupa moze imati vise od 1 realizacije, izomorfizam
\section{Симетрична група $S_n$, пермутације}
Definicija permutacije, ciklusi, parnost.
Definicija simetricne grupe
cayleyeva teorema
\section{Лагранжова теорема}
\section{Инваријантне подгрупе, фактор група}
Konjugacija, unutrasnji automorfizam, invarijantna (normalna) podgrupa, faktor grupa, homomorfizam.
Veza homomorfizma i normalnih podgrupa.

\chapter{Теорија група у физици}
\section{Векторски простор}
\begin{defn}
Нека је $(V, +)$ комутативна група, а $F, +, \cdot$ поље. $V$ је векторски простор над пољем $F$, ако је дефинисано пресликавање $F\times V\to V$, при шему слику пара $(\alpha, a)$ означавамо са $\alpha a$, тако да за свако $\alpha, \beta \in F$, $a, b\in V$ важи:
\begin{enumerate}
\item $\alpha(a+b) = \alpha a+\alpha b$
\item $(\alpha + \beta)a = \alpha a+ \beta a$
\item $(\alpha\beta)a = \alpha(\beta a)$
\item $1a = a$
\end{enumerate}
где је са $1$ означен неутрални елеменат за множење поља $F$
\end{defn}
\begin{defn}
У векторском простору $V(F)$, вектор $v$ је \emph{линеарна комбинација} вектора $a_1, ..., a_n$ ако постоје скалари $\alpha_1, ...\alpha_n$ такви да је $$v = \alpha_1 a_1+\cdots+\alpha_n a_n$$
\end{defn}
\begin{defn}
Неки скуп вектора $S\subset V(F)$ \emph{генерише} векторски простор $L(S)$, чији су елементи све линеарне комбинације вектора из $S$.
\end{defn}
\begin{defn}
У векторском простору $V(F)$ скуп вектора $a_1, ...,a_n$ је \emph{линеарно зависан} , ако постоје скалари $\alpha_1, ...,\alpha_n$, од који је бар један различит од нуле, такви да је $$\alpha_1 a_1 + \cdots+\alpha_n a_n = 0$$. Низ вектора који није линеарно зависан је \emph{линеарно независан}.
\end{defn}
\begin{defn}
\emph{База} векторског простора је низ вектора који је линеарно независан и који генерише векторски простор.
\end{defn}
Може се показати да постоји још еквивалентних дефиниција, односно потребних и довољних услова да би неки низ вектора био база. Наводимо их овде, јер ће бити корисне у далјем тексту:
\begin{itemize}
  \item У векторском простору $V(F)$ низ вектора је база ако и само ако је тај низ максималан линеарно независан низ.
  \item У векторском простору $V(F)$ низ вектора је база ако и само ако је тај низ минималан низ генератора
  \item У векторском простору $V(F)$ низ вектора $a_1, ..., a_n$ је база ако и само ако се сваки вектор $x\in V$ може на јединствен нечин написати у облику $$x = \sum_1^n\alpha_i a_i, \quad \alpha_1, ..., \alpha_n\in F$$
\end{itemize}
Одавде се може наслутити да све базе неког одређеног векторског простора $V(F)$ имају исти број елемената, што се може и показати, а тај број се зове димензија векторског простора, и обележава се са $\operatorname{dim} V$.\\
Сад кад смо дефинисали неке основне појмове везане за векторске просторе, можемо наставити са дефинисањем линеарних трансформација и њихове везе са матрицама.
\begin{defn}
Нека су $V_1$ и $V_2$ векторски простори над истом пољем $F$. Пресликавање $A: V_1\to V_2$ такво да је
$$(\forall a, b\in V_1)(\forall \alpha, \beta \in F) A(\alpha a+ \beta b) = \alpha A(a) + \beta A(b)$$
назива се \emph{линеарна трансформација (линеарни оператор, хомоморфизам)} векторског простора $V_1$ у $V_2$. Уколико је $V_1 = V_2$, тада је $A$ просто линеарна трансформација векторског простора $V$.
\end{defn}
Значајно је да је свака линеарна трансформација на јединствен начин одређена ако је познато како се пресликавају вектори базе, што нам омогућава да представимо произвољну линеарну трансформацију неког векторског простора помоћу матрице, на следећи начин:\\
Претпоставимо да је $(a_1, ...a_n)$ база векторског простора $V(F)$. Тада, линеарну трансформацију $A$ можемо задати са
$$A(a_1) = b_1, A(a_2) = b_2, ..., A(a_n) = b_n$$
где су $b_1, ..., b_n\in V$. Пошто је $(a_1, ...a_n)$ база, сваки од вектора $b_i$ може на јединствен начин написати као линеарна комбинација вектора базе, па имамо:
$$A(a_1) = \alpha_{11}a_1 + \alpha_{21}a_2 + ... + \alpha_{n1}a_n$$
$$A(a_2) = \alpha_{12}a_1 + \alpha_{22}a_2 + ... + \alpha_{n2}a_n$$
$$...$$
$$A(a_n) = \alpha_{1n}a_1 + \alpha_{2n}a_2 + ... + \alpha_{nn}a_n$$
Ове коефицијенте можемо записати и у компактнијем облику, као матрицу:
$$[A] = \begin{bmatrix}
    \alpha_{11} & \alpha_{12} & ... & \alpha_{1n} \\
    \alpha_{21} & \alpha_{22} & ... & \alpha_{2n} \\
    ... & ... & ... & ... \\
    \alpha_{n1} & \alpha_{n2} & ... & \alpha_{nn} 
  \end{bmatrix},$$
па вредност функције $A(x)$ (као линеарне трансформације произвољног вектора $x\in V$) можемо израчунати простим множењем матрица:
$$[A][x] = \begin{bmatrix}
            \alpha_{11} & \alpha_{12} & ... & \alpha_{1n} \\
            \alpha_{21} & \alpha_{22} & ... & \alpha_{2n} \\
            ... & ... & ... & ... \\
            \alpha_{n1} & \alpha_{n2} & ... & \alpha_{nn}
           \end{bmatrix}
           \begin{bmatrix}
           \zeta_1\\
           \zeta_2\\
           \vdots \\
           \zeta_n
           \end{bmatrix}
$$
при чему је $x = \zeta_1 a_1 + \zeta_2 a_2 + ... \zeta_n a_n$. Слично, може се показати да се и композиција линеарних трансформација може представити множењем одговарајућих матрица (које морају да се односе на исту базу).

Linearne transformacije, veza sa matricama
Skalarni proizvod (unitaran vektorski prostor).
Ortogonalnost, cuvanje skalarnog proizvoda.
\section{Изометријске трансформације $n$-димензионог простора}
Moze biti i nad R
GL(N, C) - opsta linearna grupa
SL(N, C) - specijalna linearna grupa
U(N, C) - unitarna grupa
SU(N, C) - specijalna unitarna grupa
O(N, C) - ortogonalna grupa
SO(N, C) - specijalna ortogonalna
\section{Дводимензионалне тачкасте кристалографске групе}
\begin{defn}[Мрежа]
\v Sta je mre\v za...
\end{defn}
\begin{defn}
Група $G\subseteq O(n, \mathbb{R})$ у односу на коју је мрежа реда $n$ инваријантна се зове \emph{кристалографска тачкаста група}, ако
\end{defn}

\begin{thm}
Свака кристалографска тачкаста група у две димензије је коначна
\end{thm}
\begin{proof}
Једноставном провером добијамо да се свака изометрија која чува неку тачку $O$ може представити као ротација око тачке $O$, осна симетрија у односу на неку праву која пролази кроз $O$, или као композиција неке такве ротације и осне симетрије. Јасно је да је довољно показати да имамо коначно много ротација, јер додавањем осних симетрија у (коначну) групу $C_n$, реда $n$, добијамо такође коначну групу $D_n$, реда $2n$.\\
Дакле, претпоставимо супротно, да је група $G$ кристалографска тачкаста група у све димензије, и садржи бесконачно много ротација. Пошто је свака ротација одређена са једним реалним бројем из $[0,2\pi)$, за свако $\epsilon>0$ по Дирихлеовом принципу можемо наћи $f, g\in G$, такве да је $f\neq g$ и $|f-g|\leq\epsilon$. Без умањења општости, претпоставимо да је $f>g$. Пошто је $G$ група, и $f-g$ припада $G$, односно $G$ садржи ротацију за произвољно мали угао. То је контрадикција са дискретношћу мреже коју очувава дата група $G$ (јер добијамо да можемо наћи произвољно блиске тачке у тој мрежи).
\end{proof}
\begin{cor}
У две димензије тачкасте кристалографске групе могу бити само облика $C_n$ и $D_n$, за $n\in\mathbb{N}$
\end{cor}

\begin{thm}
За дату кристалографску групу $G$, која чува решетку у $\mathbb{R}^2$ или $\mathbb{R}^3$, њена подгрупа ротација $H$ може бити искључиво реда 1, 2, 3, 4, или 6.
\end{thm}
\begin{proof}
Директном конструкцијом добијамо да постоје решетке у две и три димензије, које су инваријантне при свакој од ротација за $2\pi/n$, за $n\in{1, 2, 3, 4, 6}$. Остаје да се покаже да је $n\neq 5$ и $n\leq 6$.\\
Уочимо неку тачку $A$ те решетке, и посматрајмо њена растојања од свих осталих тачака те решетке. Пошто је решетка дискретна, постоји тачка $B$, која је најближа тачки $A$, на удаљености $d = |AB|$.\\
У случају да је $n>6$, посматрамо троугао $\triangle ABC$, при чему је $C$ тачка решетке настала ротирањем тачке $B$ око $A$, за угао $\varphi = 2\pi/n$. По косинусној теореми, $|BC|^2 = 2 d^2-2 d^2 \cos\varphi$, односно $|BC| = d\sqrt{2(1-\cos\varphi)}$. За $n=6$ добијамо да је $|BC| = d$, а за све остале $n>6$ важи да је $\cos(2\pi/n)>\frac12$, односно $|BC|<d$, што је контрадикција са минималношћу $d$.\\
Дакле, остаје случај $n=5$. Тада уочимо правилни петоугао странице $d$ чија су темена на решетци. Због особина групе $G$ и решетке, ивице тог петоугла можемо пресложити у петокраку звезду, тако да њена темена и даље буду на решетци. Наравно, темена те звезде су и даље темена неког правилног петоугла, али са мањом страницом $d'<\frac d 2$. Овај поступак можемо понављати произвољан број пута, и тако добити тачке те решетке на произвољно малој удаљености, што је контрадикција са дискретношћу решетке.
\end{proof}
\section{Lorentz-ова група}
Постоји један интересантан хомоморфизам између $SL(2, \mathbb{C})$ и Lorentz-ове групе $L$, групе свих линеарних трансформација векторског простора $\mathbb{R}^4$ које чувају Lorentz-ову метрику
$$|x|:=x_0^2-x_1^2-x_2^2-x_3^2.$$
Сваком вектору $x\in\mathbb{R}^4$ придружимо једну $2\times 2$ матрицу $\psi(x)$, на следећи начин:
$$\psi(x) = \begin{bmatrix}
                x_0+x_3 & x_1-ix_2 \\
                x_1+ix_2 & x_0-x_3
            \end{bmatrix}, $$
тако да важи $|x| = det(\psi(x))$. Тада, пресликавање $\varphi: SL(2, \mathbb{C}) \to L$, дато са $\varphi(A)(x) = \psi^{-1}(A\psi(x)A^*)$ је хомоморфизам, при чему се матрица $A*$ добија транспоновањем матрице $A$ и коњуговањем свих њених елемената. И заиста, $\psi$ је линеарни изоморфизам из $\mathbb{R}^4$ у
$$\psi(\mathbb{R}^4) = \left\lbrace\left. \begin{bmatrix} x & y \\ z&u \end{bmatrix}\right|x, y, z, u\in \mathbb{C} \text{ i } \overline x = x, \overline y = z, \overline u = u \right\rbrace$$
Може се проверити да је овај простор инваријантан под $M\to AMA^*$, за свако $A\in GL(2, \mathbb{C})$. Такође, за $A\in SL(2, \mathbb{C})$, $\varphi(A)$ чува метрику, због мултипликативних својстава детерминанте:
$$|\varphi(A)(x)| = \det(\psi(\varphi(A)(x))) = \det(A\psi(x)A^*) = \det(A)\det(\psi(x))\det(A^*) = \det(\psi(x)) = |x|$$
\chapter{Закључак}

\bibliography{literatura}

\end{document}
