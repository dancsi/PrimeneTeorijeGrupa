\documentclass{report}
\usepackage[a4paper]{geometry}


\usepackage{graphicx}
\usepackage{booktabs}
\usepackage{amsthm, amsmath, amssymb, amsfonts}
\usepackage{float, subcaption}
\usepackage{datetime}

\usepackage{fontspec}
\usepackage{polyglossia}
\setmainlanguage[Script=Cyrillic]{serbian}
\setotherlanguage{english}
\setmainfont{Times New Roman}

\theoremstyle{plain}
\newtheorem{thm}{Теорема}
\newtheorem*{cor}{Последица}
\theoremstyle{definition}
\newtheorem{defn}{Дефиниција}

\begin{document}

\title{Примене Теорије Група}

\author{Даниел Силађи}
\date{\today, \currenttime}

\maketitle

\tableofcontents

\chapter{Мотивација, групе и симетрије}
У свакодневном животу се често дешава да се сусрећемо са предметима за које
кажемо да су ''симетрични''. Шта је заправо симетрија? На пример, можемо рећи да је
неки објекат симетричан ако ''изгледа исто'' кад га гледамо са различитих тачака гледишта.\\
Формалније, симетрија неког објекта је нека трансформација у простору која пресликава
тај објекат на њега самог. Временом, људи су приметили да све овакве трансформације
имају још нека својства, карактеристична за све трансформације:
\begin{enumerate}
  \item Трансформације симетрије су бијективне, тј. за сваку трансформацију постоји инверзна трансформација која ''поништава'' њен ефекат, и доводи објекат у почетно стање. На пример, у случају ротације за угао $\varphi$ око неке осе, инверзна трансформација је ротација за угао $-\varphi$, односно $2\pi-\varphi$ око исте те осе. Неке трансформације, попут осне симетрије могу бити и саме себи инверзне.
  \item Комбиновањем (композицијом) две трансформације такође добијамо трансформацију. На пример, композиција две ротације (са заједночком осом ротације) је опет ротација, композиција две симетрије може бити транслација или ротација, ...
  \item Сваки објекат има једну (тривијалну) симетрију, идентичко пресликавање, које слика сваку тачку у њу саму.
\end{enumerate}
Касније ћемо видети да скуп трансформација (и уопште било каквих апстрактних математичких објеката) са оваквим својствима чини \emph{групу} (групу трансформација у овом случају).
\section{Симетрије раванских фигура}
За почетак, кренимо од неких једноставнијих фигура, правилних многоуглова. Опет, од њих је најједноставнији (једнакостранични) троугао.\\
\begin{figure}[h]
\centering
\includegraphics[width=0.5\textwidth]{placeholder}
\caption{Симетрије једнакостраничног троугла}
\end{figure}\\
Као што видимо, постоји 6 трансформација симетрије: \begin{itemize}
                                                         \item Ротације око центра троугла: $r$ (за $\pi/3$), $r\circ r = r^2$ (за $2\pi/3$) и $r\circ r \circ r = r^3 = e$ (за $2\pi$, односно $0$ радијана - идентичко пресликавање)
                                                         \item Осне симетрије $s_1$, $s_2$ i $s_3$ у односу на праве $\ell_1$, $\ell_2$, $\ell_3$. Приметимо да опет важи $s_i \circ s_i = s_i^2 = e$, за $i\in \lbrace 1, 2, 3 \rbrace$, али и $s_1 s_2 = r^2$, ...
                                                       \end{itemize}
Слично се дешава и код квадрата и правилног шестоугла, само са више оса симетрије и већим степеном ротационе симетрије:\\
\begin{figure}[h]
\centering
\begin{subfigure}{0.4\textwidth}
\includegraphics[width=\textwidth]{placeholder}
\end{subfigure}
~
\begin{subfigure}{0.4\textwidth}
\includegraphics[width=\textwidth]{placeholder}
\end{subfigure}
\caption{Simetrije kvadrata i pravouglog \v sestougla}
\end{figure}\\
Са друге стране, постоје и примери код којих се јавља и транслациона симетрија. Јасно је да то морају бити бесконачни цртежи код којих се један или више основних елементата понављају по једној (\emph{frieze}) или две димензије (\emph{wallpaper}). Интересантно је да се обе врсте оваквих цртежа могу потпуно класификовати у зависности од симетрија које поседују. Тако имамо $7$ frieze-група и $17$ wallpaper-група трансформација.
\begin{figure}[h]
\centering
\begin{subfigure}{0.4\textwidth}
\includegraphics[width=\textwidth]{placeholder}
\end{subfigure}
~
\begin{subfigure}{0.4\textwidth}
\includegraphics[width=\textwidth]{placeholder}
\end{subfigure}
\caption{Примери frieze цртежа}
\end{figure}\\

\begin{figure}[h]
\centering
\begin{subfigure}{0.4\textwidth}
\includegraphics[width=\textwidth]{placeholder}
\end{subfigure}
~
\begin{subfigure}{0.4\textwidth}
\includegraphics[width=\textwidth]{placeholder}
\end{subfigure}
\caption{Примери wallpaper-а}
\end{figure}

\section{Симетрије тродимензионалних тела}
У три димензије, добро су познате симетрије правилних полиедара (тетраедра, коцке, октаедра, икосаедра и додекаедра), али су значајне и групе симетрија молекула и кристала (кристалографске групе), које се такође могу комплетно класификовати, као у дводимензионалном случају.

\begin{figure}[h]
\centering
\begin{subfigure}{0.3\textwidth}
\includegraphics[width=\textwidth]{placeholder}
\caption{Молекул}
\end{subfigure}
~
\begin{subfigure}{0.3\textwidth}
\includegraphics[width=\textwidth]{placeholder}
\caption{Молекул}
\end{subfigure}
~
\begin{subfigure}{0.3\textwidth}
\includegraphics[width=\textwidth]{placeholder}
\caption{Кристална решетка}
\end{subfigure}
\end{figure}

\section{Симетрије у физици}
%Simetrije u klasicnoj i kvantnoj mehanici, npr lorencova grupa i resenja sredingerove jednacine za vodonikov atom. Lijeve grupe.
Коначно, симетрије нашег простора (Poincaré-ова и Lorentz-ова група у специјалној теорији релативности) произилазе из свих квантних теорија поља, док је сам Стандардни Модел (најприхваћенија таква теорија) базиран на унутрашњим симетријама групе $SU(3)\times SU(2)\times U(1)$ (касније ће бити бити објашњено шта то значи).

\chapter{Увод у теорију група}
\section{Основне дефиниције}
%Definicija grupe, red grupe, red elementa, podgrupe.
Као што је већ било речено, елементи неке групе не морају нужно да буду трансформације, већ елементи произвољног скупа $G$, за које смо дефинисали операцију ''множења'' (формално: операцију групе) која задовољава следеће особине:
\begin{enumerate}
  \item Ако $a$ и $b$ припадају $G$, онда и њихов производ, $ab$ припада $G$
  \item Операција множења је асоцијативна, односно важи $a(bc) = (ab)c$
  \item $G$ садржи \emph{јединични елемент} $e$, за који важи $ae = ea = a$, за свако $a\in G$
  \item За свако $a\in G$ постоји $b\in G$, за које важи $ab = ba = e$. Такав $b$ се зове \emph{инверзни елемент} за $a$, и обележава се са $a^{-1}$
\end{enumerate}
У овом тренутку је корисно доказати да је јединични елемент јединствен, као и да за сваки елемент групе постоји јединствен инверзни елемент (иначе ознака $a^{-1}$ не би имала смисла). Претпоставимо дакле супротно, да постоје бар два различита јединична елемента, $e_1$ и $e_2$. Са једне стране, њихов производ је $e_1 e_2 = e_1$, а са друге стране, $e_1 e_2 = e_2$. Ово је контрадикција, дакле, постоји јединствен јединични елемент.

Докажимо сада да је инверзни елемент јединствен. Опет, претпоставимо супротно, да за неки елемент $a$ постоје бар два различита инверзна елемента, $b$ и $b'$. Али, када то распишемо,
$$b = be = b(ab') = (ba)b' = eb' = b',$$
опет долазимо до контрадикције, јер смо претпоставили да је $b\neq b'$. Дакле, за сваки елемент неке групе постоји јединствен њему инверзни елемент.

Иако операцију групе често називамо ''множењем'', она заправо и може а и не мора то да буде:
\begin{itemize}
  \item Скуп рационалних (или реалних) бројева без $0$ чини групу у односу на множење (у уобичајеном смислу)
  \item Скуп целих (али не и природних!) бројева чини групу у односу на сабирање
  \item Скуп свих инвертибилних (регуларних, њихова детерминанта је различита од $0$) квадратних матрица, над пољем $\mathbb{R}$ или $\mathbb{C}$, димензија $n\times n$, чини групу, а операција групе је множење матрица.
\end{itemize}
Али, чак и у овим примерима нисмо причали о апстрактим групама, него о њиховим конкретним примерима, реализацијама. Структура неке апстрактне групе је задата искључиво дефинисањем операције множења сваког уређеног пара елемената, било набрајањем или на неки други начин, али без позивања на ''природу'' тих елемената.\\
Приметимо да операција множења не мора бити комутативна (нпр код множења матрица), али ако јесте, односно ако за свако $a, b\in G$ важи $ab = ba$, онда је група $G$ комутативна или \emph{Абелова}.\\
Број елемената у групи назива се \emph{ред групе}.\\
Ако одаберено неки елемент $a$ групе $G$, можемо га помножити са самим собом и добити производ $aa$, који ћемо обележавати са $a^2$. У општем случају, производ $$\underbrace{aa...a}_{n\text{ пута}}$$ обележавамо са $a^n$. Слично, можемо дефинисати и негативне степене $a$:
$$a^{-n} = (a^{-1})^n = (a^n)^{-1}$$
Ако су сви степени $a$ различити, кажемо да је $a$ \emph{бесконачног реда}. Иначе, исписивањем степена $a$, наићи ћемо на два природна броја $r$ и $s$, $r>s$, за које важи
$$a^r = a^s.$$
Множењем обе стране једнакости са $a^{-s}$, добијамо
$$a^{r-s} = a^0 = e, r-s>0.$$
Нека је $n$ најмањи број за који је $a^n = e$, $n>0$. Тада је $n$ \emph{ред елемента $a$}.

Непразан подскуп $H$ групе $G$ је \emph{подгрупа} групе $G$, ако је $H$ група у односу на рестрикцију операције групе $G$ на скупу $H$. Свака група $G$ са јединичним елементон $e$ има \emph{тривијалне подгрупе} $\{e\}$ и $G$.

Нека је $A$ непразан подскуп групе $G$. Обележимо са $[A]$ пресек свих подгрупа $G$ које садже скуп $A$. Пошто се може показати да је пресек две подгрупе неке групе и даље подгрупа те групе, и $[A]$ је подгрупа групе $G$. Јасно је да је $[A]$ најмања подгрупа $G$ која садржи скуп $A$.

Нека је $\emptyset \neq A\subset G$. Тада подгрупу $[A]$ зовемо \emph{подгрупа генерисана скупом $A$}. Ако је $[A]=G$, онда скуп $A$ називамо \emph{генераторни скуп} групе $G$, а његове елементе \emph{генератори} групе $G$.


\section{Групе реда 1, 2, 3, 4}

Сада, када смо дефинисали неке основне појмове из теорије група, можемо класификовати све групе реда 1, 2, 3 и 4, као пример.

Група реда 1 садржи само један елемент, и то мора бити јединични елемент $e$, за који важи $ee=e$.

Група реда 2 такође садржи јединични елемент $е$, и још један елемент $a\neq e$. Остаје да видимо чему је једнако $aa$, односно $a^2$. Пошто група има само 2 елемента, или је $a^2 = a$ или је $a^2=e$. У првом случају долазимо до контрадикције, јер би добили да је $a=e$. Дакле, остаје $a^2=e$, односно $a=a^{-1}$. Примери ове групе су
\begin{itemize}
  \item Бројеви 0 и 1, при чему је операција групе сабирање по модулу 2.
  \item Бројеви 1 и -1, при чему је операција групе множење
  \item Јединична трансформација ($e$) и раванска симетрија тродимензионалног еуклидског простора ($a$). Операција групе је композиција две трансформације
  \item Слично као претходно, само што је $a$ централна симетрија у односу на координатни почетак
  \item Слично као претходно, само што је $a$ ротација око неке осе, на пример $z$, за $180^\circ$
\end{itemize}
Као што се види, све ове групе имају у основи исту структуру, и за њих кажемо да су \emph{изоморфне}.

\begin{defn}
У општем случају, кажемо да су две групе $G$ и $H$ изоморфне, ако постоји бијективно пресликавање $\varphi: G\to H$, које има особину да за свако $a, b\in G$ важи $$\varphi(ab) = \varphi(a)\varphi(b)$$
Приметимо овде да производ $ab$ рачунамо у групи $G$, а производ $\varphi(a)\varphi(b)$ према правилима групе $H$.
\end{defn}

Посматрајмо сада групу реда 3, нека су њени елементи $e, a, b$. Слично као у случају групе реда 2, производи $ab$ и $ba$ не могу бити ни $a$ ни $b$, јер би тада било $a=e$ или $b=e$. Дакле, $ab = ba = e$. Елемент $a^2$ не може бити ни $a$ ни $e$, него само $b$. Тако добијамо да су елементи ове групе заправо $a$, $a^2$ и $a^3 = e$. То је пример \emph{цикличне групе} (реда 3 у овом случају), која је генерисана само једним елементом. Таква група постоји за произвољан природан број $n$, обележава се са $C_n$ (циклична група реда $n$), и њени елементи су $\{a, a^2, ..., a^n = e\}$.

Да бисмо олакшали себи посао око представљања неке групе, уобичајено је њене елементе записати у (Кејлијеву) таблицу групе, својеврсне таблице множења за ту групу. На пример, за $C_3$ имамо:\\

\begin{tabular}{c|c c c}
      & $e$ & $a$ & $b(=a^2)$ \\ \hline
  $e$ & $e$ & $a$ & $b$ \\
  $a$ & $a$ & $b$ & $e$ \\
  $b$ & $b$ & $e$ & $a$ \\
\end{tabular}\\

Приметимо да су сви елементи у једној колони и у једној врсти различити, што директно произилази из својства групе (јер би у супротном, за $x\neq y$ имали $ax = ay \Rightarrow a^{-1}ax = a^{-1}ay \Rightarrow x=y$).\\

Коначно, сличним резоновањем долазимо до закључка да постоје две различите, неизоморфне, групе реда 4:
\begin{enumerate}
  \item Циклична група реда 4, $C_4$ :
  \begin{tabular}{c|c c c c}
      & $e$ & $a$ & $b$ & $c$ \\ \hline
  $e$ & $e$ & $a$ & $b$ & $c$ \\
  $a$ & $a$ & $b$ & $c$ & $e$ \\
  $b$ & $b$ & $c$ & $e$ & $a$ \\
  $c$ & $c$ & $e$ & $a$ & $b$ \\
  \end{tabular}
  \item Клајнова 4-група, $V$ :
  \begin{tabular}{c|c c c c}
      & $e$ & $a$ & $b$ & $c$ \\ \hline
  $e$ & $e$ & $a$ & $b$ & $c$ \\
  $a$ & $a$ & $e$ & $c$ & $b$ \\
  $b$ & $b$ & $c$ & $e$ & $a$ \\
  $c$ & $c$ & $b$ & $a$ & $e$ \\
  \end{tabular}
\end{enumerate}

\section{Симетрична група $S_n$, пермутације}
Познато је да су пермутације скупа ${1, 2, ..., n}$ све функције $\pi$ које бијективно пресликавају тај скуп у самог себе. Уобичајено је да се пермутација записује на следећи начин:
$$\pi = \begin{pmatrix}
            1 & 2 & ... & n \\
            a_1 & a_2 & ... & a_n
        \end{pmatrix}$$
при чему је $\{a_1, a_2, ..., a_n\} = \{1, 2, ..., n\}$. Пошто је композиција две бијекције (пермутације) бијекција (пермутација), и пошто за свако пермутацију постоји њој инверзна пермутација, можемо закључити да све ове пермутације једног скупа чине групу (у односу на операцију композиције ових функција-пермутација), и та група се назива \emph{симетрична група скупа од $n$ елемената}, и обележава се са $S_n$.

Са друге стране, посматрајмо неку конкретну пермутацију, на пример:
$$\pi = \begin{pmatrix}
    1 & 2 & 3 & 4 & 5 & 6 & 7 & 8 \\
    2 & 3 & 1 & 5 & 4 & 7 & 6 & 8
  \end{pmatrix},$$
видимо да се 1 слика у 2, затим 2 у 3, и 3 опет у 1. Ови бројеви на тај начин формирају \emph{циклус}, који записујемо као $(123)$. Слично, 4 и 5 формирају циклус $(45)$, 6 и 7 формирају циклус (67), и број 8 формира циклус од једног елемента, $(8)$. Сада ову пермутацију можемо да запишемо на алтернативан начин, као:
$$(123)(45)(67)(8)$$
Приметимо да су сви ови циклуси дисјунктни. Даље, записивање ове пермутације као неког ''производа'' заиста има смисла, ако саме циклусе посматрамо као пермутације, а њихов ''производ'' као композицију тих пермутација. И заиста, за циклус $(123)$ имамо пермутацију
$$\begin{pmatrix}
    1 & 2 & 3 & 4 & 5 & 6 & 7 & 8 \\
    2 & 3 & 1 & 4 & 5 & 6 & 7 & 8
  \end{pmatrix},$$
за циклусе $(45)$ и $(67)$ пермутације
$$\begin{pmatrix}
    1 & 2 & 3 & 4 & 5 & 6 & 7 & 8 \\
    1 & 2 & 3 & 5 & 4 & 6 & 7 & 8
  \end{pmatrix} \text{ и }
  \begin{pmatrix}
    1 & 2 & 3 & 4 & 5 & 6 & 7 & 8 \\
    1 & 2 & 3 & 4 & 5 & 7 & 6 & 8
  \end{pmatrix},$$
и за циклус $(8)$ идентичну пермутацију
$$\begin{pmatrix}
    1 & 2 & 3 & 4 & 5 & 6 & 7 & 8 \\
    1 & 2 & 3 & 4 & 5 & 6 & 7 & 8
  \end{pmatrix}.$$
Приметимо пар чињеница:
\begin{itemize}
  \item Композиција ових пермутација заиста даје почетну пермутацију, $\pi$
  \item Редослед записивања (дисјунктних) циклуса није битан, односно
  $$(123)(45) = (45)(123)$$
  \item У појединачним циклусима, можемо узети било који елемент као почетни,
  $$(123) = (231) = (312)$$
  \item Циклус $(8)$, односно идентичка пермутација се може изоставити, само треба водити рачуна о броју елемената у тој пермутацији
  \item Број пермутованих елемената је 7, а број независних циклуса (не рачунајући циклусе од једног елемента) је 4. Разлика ова два броја је \emph{декремент пермутације}. Дефинишемо \emph{парност пермутације} као парност декремента.
\end{itemize}

Једна од основних теорема у теорији група је Кејлијева (Cayley) теорема, која гласи овако:
\begin{thm}
Свака група $G$ реда $n$ је изоморфна са подгрупом симетричне групе $S_n$.
\end{thm}
\begin{proof}
Нека је $G = \{a_1, a_2, ..., a_n\}$ . Посматрајмо произвољни елемент $b\in G$, и његове производе $ba_1, ba_2, ..., ba_n$ са свим осталим елементима $G$. Као што смо раније приметили, сви ови производи морају бити различити. Зато, производи $ba_i$ су у ствари нека пермутација скупа $G$:
$$b\to \pi_b = \begin{pmatrix}
                a_1 & ... & a_n \\
                ba_1 & ... & ba_n
               \end{pmatrix}$$
Сличне пермутације можемо придружити и осталим елементима $G$. Пошто желимо да покажемо да постоји изоморфизам између групе $G$ и групе ових пермутација, треба да покажемо да ће за произвољно $b, c\in G$ пермутација $\pi_c \pi_b$ заиста одговарати елементу $cb$. Посматрајмо дакле пермутацију $\pi_c$:
$$\pi_c = \begin{pmatrix}
                a_1 & ... & a_n \\
                ca_1 & ... & ca_n
          \end{pmatrix}$$
Она се може написати и на други начин, као
$$\pi_c = \begin{pmatrix}
                ba_1 & ... & ba_n \\
                c(ba_1) & ... & c(ba_n)
          \end{pmatrix}$$
Сада лако можемо добити производ $\pi_c \pi_b$:
$$\pi_c \pi_b= \begin{pmatrix}
                ba_1 & ... & ba_n \\
                c(ba_1) & ... & c(ba_n)
               \end{pmatrix}
               \begin{pmatrix}
                a_1 & ... & a_n \\
                ba_1 & ... & ba_n
               \end{pmatrix}$$
Множењем ове две пермутације добијамо
$$\pi_c \pi_b= \begin{pmatrix}
                a_1 & ... & a_n \\
                cba_1 & ... & cba_n
               \end{pmatrix} = \pi_{cb}$$

Дакле, тражени изоморфизам $\varphi: G\to S_n$ је $\varphi(a) = \pi_a$.
\end{proof}
Ова теорема нам показује да постоји коначан број група реда $n$, и даје нам систематичан начин за налажење сви тих група (све подгрупе $S_n$).

\section{Лагранжова теорема}
Још једна теорема корисна за одређивање структура свих модућих група датог реда, али и за многе друге проблеме, је Лагранжова (Lagrange) теорема.
\begin{thm}
Ако је $G$ нека група реда $n$, и $H$ њена подгрупа реда $m$, тада $m$ дели $n$.
\end{thm}
\begin{proof}
Ако је $H=G$, тврђење је тривијално тачно. Иначе, постоји неки елемент $a\in G$, тако да $a\notin H$. Ако обележимо елементе подгрупе $H$ са $e, b_2, ..., b_m$, тада дефинишемо скуп $aH$ као скуп производа $a$ и свих елемената групе $H$,
$$aH = \{a, ah_2, ..., ah_m\}.$$
Због особина групе имамо да су сви $ah_i$ различити (иначе би постојало $j, j\neq i$, за које је $h_i=h_j$), као и да је $aH\cap H = \emptyset$ (иначе би било и $a\in H$).

Сада имамо два скупа од по $m$ различитих елемената, $H$ и $aH$, који се садрже у $G$. Уколико у $G$ joш постоје елементи који нису ни у $H$, ни у $aH$, понављамо овај поступак за један такав елемент, $c$, и формирамо $cH$. Као резултат, поделили смо групу $G$ на $k$ дисјунктних подскупова:
$$G = H \cup a_1 H \cup a_2 H \cup ... \cup a_{k-1} H.$$
Број $k$ се зове \emph{индекс} подгрупе $H$ у групи $G$. Скупови $a_i H$ се зову \emph{леве класе} $H$ у $G$, а скуп свих левих класа неке подгрупе се обележава са $G/H$. Приметимо да смо на сличан начин могли да поделимо $G$ и на \emph{десне класе}:
$$G = H \cup Ha_1'  \cup Ha_2' \cup ... \cup Ha_{k-1}'.$$
\end{proof}
\begin{cor}
Ред елемента (цикличне подгрупе генерисане тим елементом) дели ред групе.
\end{cor}

\section{Инваријантне подгрупе, фактор група}
%Konjugacija, unutrasnji automorfizam, invarijantna (normalna) podgrupa, faktor grupa, homomorfizam.
\subsection{Класе конјугације}
\begin{defn}
За елемент $b$ групе $G$ кажемо да је \emph{конјугован} елементу $a$ ако постоји $u\in G$ за који важи
$$u a u^{-1} = b$$
\end{defn}
Видимо да ако одаберемо $u = e$, добијамо да је $a$ само себи конјуговано. Такође, ако је $b$ конјуговано са $a$, $c$ конјуговано са $b$, тада је и $c$ конјуговано са $a$ (јер је $c = vbv^{-1}$, $b = uau^{-1}$, па је $c = vuau^{-1}v^{-1}$, односно $c = (vu)a(vu)^{-1}$). Коначно, једноставно се види да ако је $u a u^{-1} = b$, тада је и $a = u^{-1} b (u^{-1})^{-1}$.

На овај начин смо утврдили да је релација конјугације рефлексивна, симетрична и транзитивна, дакле - релација еквиваленције. Таква релација разбија скупа на класе еквиваленције, у овм случају \emph{класе конјугације}. Ако је група комутативна, тада сваки елемент чини класу за себе, јер за свако $a, b$, $bab^{-1} = a$.

У случају група трансформација, класе конјугације имају једноставну ''физичку'' интерпретацију, у смислу да групишу ''сличне'' трансформације у једну класу. На пример, ако са $r$ обележимо неку ротацију, а са $s$ неку осну симетрију са осом која пролази кроз центар ротације $r$, тада је $r s r^{-1}$ трансформација осне симетрије у односу на неку нову осу, насталу ротацијом осе $s$ за $r$.

\subsection{Инваријантне подгрупе}
Посматрајмо подгрупу $H$ групе $G$. Тада се лако показује да је и $aHa^{-1}$ (сви производи $aha^{-1}$, где $h\in H$) подгрупа од $G$, при чему је $a$ произвољни елемент из $G$. Уколико за свако $a\in G$ важи
$$aHa^{-1} = H,$$
тада кажемо да је $H$ \emph{инваријантна} или \emph{нормална подгрупа} групе $G$, и то обележавамо са $H\lhd G$. Последња једнакост у ствари значи да за дати елемент $h_1\in H$, за свако $a\in G$, можемо наћи $h_2\in H$, тако да буде $ah_1a^{-1} = h_2$, односно $ah_1 = h_2 a$, што се може краће записати и као
$$aH = Ha, $$
и тако добијамо алтернативну дефиницију нормалне подгрупе: Подгрупа $H$ је нормална у $G$, ако су лева и десна класа једнаке, без обзира на то са којим елементом из $G$ их формирамо.

Коначно, можемо приметити још један потребан и довољан услов да нека подгрупа буде нормална, а то је да она садржи елементе $G$ у комплетним класама конјугације, тј. садржи или све или ни један елемент из неке дате класе конјугације.

Можда ће се неко запитати чему све ово, али испоставља се да овакве подгрупе имају нека врло интересантна својства. За неку групу $G$ и њену нормалну подгрупу $H$ дефинишимо операцију $\cdot: G/H \to G/H$ као $aH\cdot bH := abH$. Прво, треба показати да резултат не зависи од избора представника класа, односно да ако важи
$$aH = a_1 H \text{ и } bH = b_1 H,$$
тада мора бити $aH\cdot bH = a_1 H \cdot b_1 H$, тј. $abH = a_1 b_1 H$.

\begin{proof}
Доказаћемо да је $abH \subseteq a_1 b_1 H$ и $a_1 b_1 H \subseteq abH$. Доказујемо прво тврђење:

Нека је $x\in abH$. Тада постоји $h\in H$, такво да $x=abh$. Пошто је $bh\in bH$, тада постоји $h_1\in H$, тако да $bh = b_1 h_1$. Дакле,
$$x = abh = ab_1 h_1.$$
Како је $H\lhd G$, $b_1 H = Hb_1$, па онда $(\exists h_2\in H)(b_1h_1=h_2b_1)$. Таквим резоновањем добијамо
$$x = ab_1h_1 = ah_2b_1 = a_1h_3b_1 = a_1b_1h_4 \in a_1b_1H.$$

Дакле $x\in abH \Rightarrow x\in a_1b_1H$, па  је $abH \subseteq a_1b_1H$. Други смер, $a_1 b_1 H \subseteq abH$ се показује аналогно, па коначно имамо $abH = a_1b_1H$.
\end{proof}

Показали смо да множењем две класе ове подгрупе опет добијамо неку класу исте те подгрупе, док је сама подгрупа $eH = H$ јединични елемент. Слично можемо наћи и инверзни елемент за ову операцију:
$$(a^{-1}H)(aH) = a^{-1}HaH = a^{-1}aHH = H.$$

Дакле, класе нормалне подгрупе и саме чине групу у односу на операцију множења класа дефинисану малопре - та група се назива \emph{фактор-група}, обележава се са $G/H$, и њен ред је индекс групе $H$ у $G$.

 %Veza homomorfizma i normalnih podgrupa.

\chapter{Теорија група у физици}
\section{Векторски простор}
За примену теорије група у физици потребно је дефинисати још и појам векторског простора (што је заправо уопштење свима добро познатог Еуклудског простора). Након тога ћемо моћи да посматрамо неке групе као групе трансформација векторских простора.
\subsection{Основне дефиниције}
Нека је $(V, +)$ комутативна група, а $(F, +, \cdot)$ поље реалних или комплексних бројева. $V$ је векторски простор над пољем $F$, ако је дефинисано пресликавање $F\times V\to V$, при шему слику пара $(\alpha, v)$ означавамо са $\alpha v$, тако да за свако $\alpha, \beta \in F$, $u, v\in V$ важи:
\begin{enumerate}
\item $\alpha(u+v) = \alpha u+\alpha v$
\item $(\alpha + \beta)v = \alpha v+ \beta v$
\item $(\alpha\beta)v = \alpha(\beta v)$
\item $1v = v$
\end{enumerate}
где је са $1$ означен неутрални елеменат за множење поља $F$. Елементи скупа $F$ се називају \emph{скалари}, а елементи скупа $V$ - \emph{вектори}.

На пример, $\mathbb{R}^n$ је векторски простор над $\mathbb{R}$, при чему је сабирање вектора дефинисано на уобичајен начин као сабирање $n$-торки.

Ова дефиниција се може уопштити и на произвољно поље, али ћемо се ми овде бавити само случајевима када је $F=\mathbb{R}$ или $F=\mathbb{C}$.

У векторском простору $V(F)$, вектор $v$ је \emph{линеарна комбинација} вектора $v_1, ..., v_n$ ако постоје скалари $\alpha_1, ...\alpha_n$ такви да је $$v = \alpha_1 v_1+ \alpha_2 v_2+ \cdots+\alpha_n v_n$$

Ако је $S\subseteq V(F)$, са $L(S)$ обележавамо скуп свих линеарних комбинација вектора из $S$. Кажемо да скуп вектора $S\subset V(F)$ \emph{генерише} векторски простор $W$, ако је $W = L(S)$.


У векторском простору $V(F)$ скуп вектора $v_1, ...,v_n$ је \emph{линеарно зависан} , ако постоје скалари $\alpha_1, ...,\alpha_n$, од који је бар један различит од нуле, такви да је $$\alpha_1 v_1 + \alpha_2 v_2 + \cdots+\alpha_n v_n = 0.$$ Низ вектора који није линеарно зависан је \emph{линеарно независан}.
\emph{База} векторског простора је низ вектора који је линеарно независан и који генерише векторски простор.
На пример, вектори $(1, 0, 0)$, $(0, 1, 0)$ и $(0, 1, 1)$ су база векторског простора $\mathbb{R}^3$.
Може се показати да постоји још еквивалентних дефиниција, односно потребних и довољних услова да би неки низ вектора био база. Наводимо их овде, јер ће бити корисне у даљем тексту:
\begin{itemize}
  \item У векторском простору $V(F)$ скуп вектора је база ако и само ако је тај скуп максималан линеарно независан скуп.
  \item У векторском простору $V(F)$ скуп вектора је база ако и само ако је тај скуп минималан скуп генератора
  \item У векторском простору $V(F)$ скуп вектора $v_1, ..., v_n$ је база ако и само ако се сваки вектор $x\in V$ може на јединствен нечин написати у облику $$x = \sum_1^n\alpha_i v_i, \quad \alpha_1, ..., \alpha_n\in F$$
\end{itemize}
Одавде се може наслутити да све базе неког одређеног векторског простора $V(F)$ имају исти број елемената, што се може и показати, а тај број се зове \emph{димензија векторског простора}, и обележава се са $\operatorname{dim} V$.

\subsection{Унитарни векторски простори}

Нека је $V$ векторски простор над пољем $F$ (где је $F=\mathbb{R}$ или $F=\mathbb{C}$). \emph{Унутрашњи (скаларни) производ} на $V$ је свака функција $(,):V\times V\to F$, при чему слику уређеног пара вектора $(x, y)\in V\times V$ означавамо са $(x, y)$, за коју за свако $x, y, z\in V$ и свако $\alpha \in F$ важи
\begin{enumerate}
  \item $(x, y) = \overline{(y, x)}$
  \item $(x+y, z) = (x, z)+(y, z)$
  \item $(\alpha x, y) = \alpha (x, y)$
  \item $(x, x)\geq 0$
  \item $(x, x) = 0 \Leftrightarrow x=0$
\end{enumerate}
Са $\overline{(x, y)}$ означен је комплексан број конјугован са $(x, y)$.

Један пример унитарног векторског простора је $\mathbb{R}^n$, при чему је за $u = (x_1, ...x_n)$ и $v = (y_1, ..., y_n)$ дефинисано $u\cdot v = \sum_{i=1}^n x_iy_i$.

Векторски простор над пољем реалних или комплексних бројева заједно са функцијом која дефинише унутрашњи производ назива се \emph{унитарни векторски простор}.
У унитарном векторском простору $V$ функција $\|\;\|:V\to \mathbb{R}$, дефинисана са $$\|x\| = \sqrt{(x, x)}$$ назива се \emph{норма} на $V$. Ненегативан реалан број $\|x\|$ назива се \emph{норма вектора} $x$.
\emph{Растојање} вектора $x$ и $y$ је дефинисано са $$d(x, y) = \|x-y\|$$

Позната Коши-Шварцова неједнакост важи у било ком унитарном векторском простору:
\begin{thm}
У унитарном векторском простору $V$ за свако $x, y\in V$ важи
$$|(x, y)|\leq\|x\|\|y\|,$$
при чему једнакост важи ако и само ако су вектори $x$ и $y$ линеарно зависни.
\end{thm}
Као последицу, (у реалним унитарним векторским просторима) можемо дефинисати угао између вектора $x\neq 0$ и $y\neq 0$, као реалан број $\alpha\in [0, \pi]$, такав да је $$\cos \alpha = \frac{(x, y)}{\|x\|\|y\|}$$
Природно, два вектора $x$ и $y$ су \emph{ортогонална} ако је $(x, y) = 0$.

\subsection{Линеарне трансформације}
Сад кад смо дефинисали неке основне појмове везане за векторске просторе, можемо наставити са дефинисањем линеарних трансформација и њихове везе са матрицама.
\begin{defn}
Нека су $V_1$ и $V_2$ векторски простори над истом пољем $F$. Пресликавање $A: V_1\to V_2$ такво да је
$$(\forall a, b\in V_1)(\forall \alpha, \beta \in F) A(\alpha a+ \beta b) = \alpha A(a) + \beta A(b)$$
назива се \emph{линеарна трансформација (линеарни оператор, хомоморфизам)} векторског простора $V_1$ у $V_2$. Уколико је $V_1 = V_2 = V$, тада је $A$ просто линеарна трансформација векторског простора $V$.
\end{defn}
Значајно је да је свака линеарна трансформација на јединствен начин одређена ако је познато како се пресликавају вектори базе, што нам омогућава да представимо произвољну линеарну трансформацију неког векторског простора помоћу матрице, на следећи начин:\\
Претпоставимо да је $\{a_1, ...a_n\}$ база векторског простора $V(F)$. Тада, линеарну трансформацију $A$ можемо задати са
$$A(a_1) = b_1, A(a_2) = b_2, ..., A(a_n) = b_n$$
где су $b_1, ..., b_n\in V$. Пошто је $\{a_1, ...a_n\}$ база, сваки од вектора $b_i$ може на јединствен начин написати као линеарна комбинација вектора базе, па имамо:
$$A(a_1) = \alpha_{11}a_1 + \alpha_{21}a_2 + ... + \alpha_{n1}a_n$$
$$A(a_2) = \alpha_{12}a_1 + \alpha_{22}a_2 + ... + \alpha_{n2}a_n$$
$$...$$
$$A(a_n) = \alpha_{1n}a_1 + \alpha_{2n}a_2 + ... + \alpha_{nn}a_n$$
Ове коефицијенте можемо записати и у компактнијем облику, као матрицу:
$$[A] = \begin{bmatrix}
    \alpha_{11} & \alpha_{12} & ... & \alpha_{1n} \\
    \alpha_{21} & \alpha_{22} & ... & \alpha_{2n} \\
    ... & ... & ... & ... \\
    \alpha_{n1} & \alpha_{n2} & ... & \alpha_{nn}
  \end{bmatrix},$$
па вредност функције $A(x)$ (као линеарне трансформације произвољног вектора $x\in V$) можемо израчунати простим множењем матрица:
$$[A][x] = \begin{bmatrix}
            \alpha_{11} & \alpha_{12} & ... & \alpha_{1n} \\
            \alpha_{21} & \alpha_{22} & ... & \alpha_{2n} \\
            ... & ... & ... & ... \\
            \alpha_{n1} & \alpha_{n2} & ... & \alpha_{nn}
           \end{bmatrix}
           \begin{bmatrix}
           \zeta_1\\
           \zeta_2\\
           \vdots \\
           \zeta_n
           \end{bmatrix}
$$
при чему је $x = \zeta_1 a_1 + \zeta_2 a_2 + ... \zeta_n a_n$. Слично, може се показати да се и композиција линеарних трансформација може представити множењем одговарајућих матрица (које морају да се односе на исту базу).

ОВДЕ ФАЛЕ ДЕФИНИЦИЈЕ ТОГА ШТА ЈЕ ИЗОМЕТРИЈА, А ШТА ЈЕ УНИТАРНА ТРАНСФОРМАЦИЈА


\section{Изометријске трансформације $n$-димензионог простора}
Moze biti i nad R
GL(N, C) - opsta linearna grupa
SL(N, C) - specijalna linearna grupa
U(N, C) - unitarna grupa
SU(N, C) - specijalna unitarna grupa
O(N, C) - ortogonalna grupa
SO(N, C) - specijalna ortogonalna
\section{Дводимензионалне тачкасте кристалографске групе}
\begin{defn}[Мрежа]
\v Sta je mre\v za...
\end{defn}
\begin{defn}
Група $G\subseteq O(n, \mathbb{R})$ у односу на коју је мрежа реда $n$ инваријантна се зове \emph{кристалографска тачкаста група}, ако
\end{defn}

\begin{thm}
Свака кристалографска тачкаста група у две димензије је коначна
\end{thm}
\begin{proof}
Једноставном провером добијамо да се свака изометрија која чува неку тачку $O$ може представити као ротација око тачке $O$, осна симетрија у односу на неку праву која пролази кроз $O$, или као композиција неке такве ротације и осне симетрије. Јасно је да је довољно показати да имамо коначно много ротација, јер додавањем осних симетрија у (коначну) групу $C_n$, реда $n$, добијамо такође коначну групу $D_n$, реда $2n$.\\
Дакле, претпоставимо супротно, да је група $G$ кристалографска тачкаста група у све димензије, и садржи бесконачно много ротација. Пошто је свака ротација одређена са једним реалним бројем из $[0,2\pi)$, за свако $\epsilon>0$ по Дирихлеовом принципу можемо наћи $f, g\in G$, такве да је $f\neq g$ и $|f-g|\leq\epsilon$. Без умањења општости, претпоставимо да је $f>g$. Пошто је $G$ група, и $f-g$ припада $G$, односно $G$ садржи ротацију за произвољно мали угао. То је контрадикција са дискретношћу мреже коју очувава дата група $G$ (јер добијамо да можемо наћи произвољно блиске тачке у тој мрежи).
\end{proof}
\begin{cor}
У две димензије тачкасте кристалографске групе могу бити само облика $C_n$ и $D_n$, за $n\in\mathbb{N}$
\end{cor}

\begin{thm}
За дату кристалографску групу $G$, која чува решетку у $\mathbb{R}^2$ или $\mathbb{R}^3$, њена подгрупа ротација $H$ може бити искључиво реда 1, 2, 3, 4, или 6.
\end{thm}
\begin{proof}
Директном конструкцијом добијамо да постоје решетке у две и три димензије, које су инваријантне при свакој од ротација за $2\pi/n$, за $n\in{1, 2, 3, 4, 6}$. Остаје да се покаже да је $n\neq 5$ и $n\leq 6$.\\
Уочимо неку тачку $A$ те решетке, и посматрајмо њена растојања од свих осталих тачака те решетке. Пошто је решетка дискретна, постоји тачка $B$, која је најближа тачки $A$, на удаљености $d = |AB|$.\\
У случају да је $n>6$, посматрамо троугао $\triangle ABC$, при чему је $C$ тачка решетке настала ротирањем тачке $B$ око $A$, за угао $\varphi = 2\pi/n$. По косинусној теореми, $|BC|^2 = 2 d^2-2 d^2 \cos\varphi$, односно $|BC| = d\sqrt{2(1-\cos\varphi)}$. За $n=6$ добијамо да је $|BC| = d$, а за све остале $n>6$ важи да је $\cos(2\pi/n)>\frac12$, односно $|BC|<d$, што је контрадикција са минималношћу $d$.\\
Дакле, остаје случај $n=5$. Тада уочимо правилни петоугао странице $d$ чија су темена на решетци. Због особина групе $G$ и решетке, ивице тог петоугла можемо пресложити у петокраку звезду, тако да њена темена и даље буду на решетци. Наравно, темена те звезде су и даље темена неког правилног петоугла, али са мањом страницом $d'<\frac d 2$. Овај поступак можемо понављати произвољан број пута, и тако добити тачке те решетке на произвољно малој удаљености, што је контрадикција са дискретношћу решетке.
\end{proof}
\section{Lorentz-ова група}

Постоји један интересантан хомоморфизам између $SL(2, \mathbb{C})$ и Lorentz-ове групе $L$, групе свих линеарних трансформација векторског простора $\mathbb{R}^4$ које чувају Lorentz-ову метрику
$$|x|:=x_0^2-x_1^2-x_2^2-x_3^2.$$
Сваком вектору $x\in\mathbb{R}^4$ придружимо једну $2\times 2$ матрицу $\psi(x)$, на следећи начин:
$$\psi(x) = \begin{bmatrix}
                x_0+x_3 & x_1-ix_2 \\
                x_1+ix_2 & x_0-x_3
            \end{bmatrix}, $$
тако да важи $|x| = det(\psi(x))$. Тада, пресликавање $\varphi: SL(2, \mathbb{C}) \to L$, дато са $\varphi(A)(x) = \psi^{-1}(A\psi(x)A^*)$ је хомоморфизам, при чему се матрица $A*$ добија транспоновањем матрице $A$ и коњуговањем свих њених елемената. И заиста, $\psi$ је линеарни изоморфизам из $\mathbb{R}^4$ у
$$\psi(\mathbb{R}^4) = \left\lbrace\left. \begin{bmatrix} x & y \\ z&u \end{bmatrix}\right|x, y, z, u\in \mathbb{C} \text{ i } \overline x = x, \overline y = z, \overline u = u \right\rbrace$$
Може се проверити да је овај простор инваријантан под $M\to AMA^*$, за свако $A\in GL(2, \mathbb{C})$. Такође, за $A\in SL(2, \mathbb{C})$, $\varphi(A)$ чува метрику, због мултипликативних својстава детерминанте:
$$|\varphi(A)(x)| = \det(\psi(\varphi(A)(x))) = \det(A\psi(x)A^*) = \det(A)\det(\psi(x))\det(A^*) = \det(\psi(x)) = |x|$$
\chapter{Закључак}

\bibliography{literatura}

\end{document}
