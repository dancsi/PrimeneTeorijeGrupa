\documentclass[times, utf8, diplomski]{fer}
\usepackage{booktabs}
\usepackage[T1]{fontenc}
\usepackage{lmodern}

\input{glyphtounicode}
\pdfgentounicode=1

\begin{document}

\title{Primene Teorije Grupa}

\author{Daniel Sila\dj i}

\maketitle

% Ispis stranice s napomenom o umetanju izvornika rada. Uklonite naredbu \izvornik ako želite izbaciti tu stranicu.
%\izvornik

% Dodavanje zahvale ili prazne stranice. Ako ne želite dodati zahvalu, naredbu ostavite radi prazne stranice.
\zahvala{}

\tableofcontents

\chapter{Motivacija, grupe i simetrije}
Symmetry registers regularity, and thus, records beauty.
A symmetry of an object or a figure in space is a transformation in space that maps the object to itself.
Sto vise simetrija, to je objekat simetricniji, dakle, lepsi.
\section{Simetrije ravanskih figura}
Trougao, kvadrat, sestougao.
Frieze, kao primer beskonacne grupe. 7 frieze grupa
Wallpaper groups, 2 primera, jedan je manje simetrican, generisan samo sa 2 translacije, i jedan koji je vise simetrican.
\section{Simetrije trodimenzionalnih tela}
Kocka, ikosaedar
Simetrije kristala/molekula, 2 primera
\section{Simetrije u fizici}
Simetrije u klasicnoj i kvantnoj mehanici, npr lorencova grupa i resenja sredingerove jednacine za vodonikov atom. Lijeve grupe.

\chapter{Uvod u teoriju grupa}
\section{Osnovne definicije}
Definicija grupe, red grupe, red elementa, podgrupe.
Primeri: brojevi (sta jeste i sta nije), transofrmacije, simetrije
Klasifikacija grupa sa 2, 3, 4 elementa, primer da 1 grupa moze imati vise od 1 realizacije, izomorfizam
\section{Simetri\v{c} cna grupa $S_n$, permutacije}
Definicija permutacije, ciklusi, parnost.
Definicija simetricne grupe
cayleyeva teorema
\section{Lagran\v{z}ova teorema}
\section{Invarijantne podgrupe, faktor grupa}
Konjugacija, unutrasnji automorfizam, invarijantna (normalna) podgrupa, faktor grupa, homomorfizam.
Veza homomorfizma i normalnih podgrupa.

\chapter{Teorija grupa u fizici}
\section{Vektorski prostor}
Definicija vektorskog prostora
Linearne transofrmacije, veza sa matricama
Skalarni proizvod (unitaran vektorski prostor).
Ortogonalnost, cuvanje skalarnog proizvoda.
\section{Izometrijske transofrmacije n-dimenzionog prostora}
Moze biti i nad R
GL(N, C) - opsta linearna grupa
SL(N, C) - specijalna linearna grupa
U(N, C) - unitarna grupa
SU(N, C) - specijalna unitarna grupa
O(N, C) - ortogonalna grupa
SO(N, C) - specijalna ortogonalna
\section{Dvodimenzionalne ta\v{c}kaste kristalografske grupe}
\section{Lorencova grupa}
U(p, q) - pseudo-unitarna grupa
O(p, q) - pseoudo-ortogonalna grupa
O(1, 3) - Lorencova grupa

\chapter{Zaključak}
Zaključak.

\bibliography{literatura}
\bibliographystyle{fer}

\end{document}
